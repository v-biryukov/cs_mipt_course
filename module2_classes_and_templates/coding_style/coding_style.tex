\documentclass{article}
\usepackage[utf8x]{inputenc}
\usepackage{ucs}
\usepackage{amsmath} 
\usepackage{amsfonts}
\usepackage{marvosym}
\usepackage{wasysym}
\usepackage{upgreek}
\usepackage[english,russian]{babel}
\usepackage{graphicx}
\usepackage{float}
\usepackage{textcomp}
\usepackage{hyperref}
\usepackage{geometry}
  \geometry{left=2cm}
  \geometry{right=1.5cm}
  \geometry{top=1cm}
  \geometry{bottom=2cm}
\usepackage{tikz}
\usepackage{ccaption}
\usepackage{multicol}
\usepackage{fancyvrb}

\usepackage{listings}
%\setlength{\columnsep}{1.5cm}
%\setlength{\columnseprule}{0.2pt}

\usepackage{colortbl,graphicx,tikz}
\definecolor{X}{rgb}{.5,.5,.5}

\definecolor{correctcolor}{RGB}{240,250,240}
\definecolor{wrongcolor}{RGB}{250,240,240}

\title{ДЗ. Работа с изображениями в формате \texttt{.ppm}}
\date{}
\begin{document}
\pagenumbering{gobble}

\lstset{
  language=C++,                % choose the language of the code
  basicstyle=\linespread{1.1}\ttfamily,
  columns=fixed,
  fontadjust=true,
  basewidth=0.5em,
  keywordstyle=\color{blue}\bfseries,
  commentstyle=\color{gray},
  stringstyle=\ttfamily\color{orange!50!black},
  showstringspaces=false,
  %numbers=false,                   % where to put the line-numbers
  numbersep=5pt,
  numberstyle=\tiny\color{black},
  numberfirstline=true,
  stepnumber=1,                   % the step between two line-numbers.        
  numbersep=10pt,                  % how far the line-numbers are from the code
  backgroundcolor=\color{white},  % choose the background color. You must add \usepackage{color}
  showstringspaces=false,         % underline spaces within strings
  captionpos=b,                   % sets the caption-position to bottom
  breaklines=true,                % sets automatic line breaking
  breakatwhitespace=true,         % sets if automatic breaks should only happen at whitespace
  xleftmargin=.2in,
  frame=tlbr,
  framesep=5pt,
  framerule=0pt,
  extendedchars=\true,
  keepspaces = true,
}
\lstset{literate=%
   *{0}{{{\color{red!20!violet}0}}}1
    {1}{{{\color{red!20!violet}1}}}1
    {2}{{{\color{red!20!violet}2}}}1
    {3}{{{\color{red!20!violet}3}}}1
    {4}{{{\color{red!20!violet}4}}}1
    {5}{{{\color{red!20!violet}5}}}1
    {6}{{{\color{red!20!violet}6}}}1
    {7}{{{\color{red!20!violet}7}}}1
    {8}{{{\color{red!20!violet}8}}}1
    {9}{{{\color{red!20!violet}9}}}1
}

\title{Coding style guide}\date{}\maketitle
В отличии от многих других языков, в языке C++ нет общепризнаного стиля написания кода. Можно основываться на рекомендациях, данных Бьёрном Страуструпом в основных рекомендациях \textit{C++ Core Guidelines}. Но и там рекомендации по стилистики не носят обязательный характер. В результате разные программисты пишут на C++ в разных стилях. В этом курсе будет использоваться стиль описанный тут.

\section{Стиль наименования идентификаторов}
Самые распространённые стили наименования идентификаторов это:
\begin{verbatim}
                 camelCase                 PascalCase              snake_case    
\end{verbatim}
В текущем семестре будет использоваться следующий стиль:
\begin{itemize}
\item \texttt{camelCase} -- для имён переменных, полей структур и классов, функций, методов, пространств имён.
\item \texttt{PascalCase} -- для имён классов, структур, перечислений и элементов перечисления.
\item \texttt{snake\_case} --  для имён файлов.
\end{itemize}


\section{Стиль отступов}
Используется стиль отступов с открывающейся скобкой на новой строке.

\begin{multicols}{2}
\begin{lstlisting}[backgroundcolor = \color{correctcolor}]
int printNumbers(int n)
{
    while (n > 0)
    {
        cout << i << " ";
        n--;
    }
}
\end{lstlisting}

\begin{lstlisting}[backgroundcolor = \color{wrongcolor}]
int printNumbers(int n) {
    while (n > 0) {
        cout << i << " ";
        n--;
    }
}
\end{lstlisting}
\end{multicols}



\section{Стиль отступов для однострочной области видимости}
В случае если тело области видимости состоит из одной строки, то скобки нужно не писать.
\begin{multicols}{2}
\begin{lstlisting}[backgroundcolor = \color{correctcolor}]
int calculateSum(int n)
{
    int sum = 0;
    for (int i = 0; i < n; ++i)
        sum += i;
        
    return sum;
}
\end{lstlisting}
\vfill
\begin{lstlisting}[backgroundcolor = \color{wrongcolor}]
int calculateSum(int n)
{
    int sum = 0;
    for (int i = 0; i < n; ++i)
    {
        sum += i;
    }
    return sum;
}
\end{lstlisting}
\end{multicols}

Исключение -- если одна из областей видимости оператора \texttt{if else} состоит из более чем одной строки, то все остальные области видимости этого операторы должны обрамляться скобочками.
\begin{multicols}{2}
\noindent
\begin{lstlisting}[backgroundcolor = \color{correctcolor}]
if (a > b)
{
    c = a;
    cout << "a is greater than b\n";
}
else
{
    c = b;
}
\end{lstlisting}
\vfill
\begin{lstlisting}[backgroundcolor = \color{wrongcolor}]
if (a > b)
{
    c = a;
    cout << "a is greater than b\n";
}
else
    c = b;
\end{lstlisting}
\end{multicols}

\section{Объявление указателей и ссылок}
При объявлении указателей звёздочка обязательно ставится рядом с названием типа. Аналогично для положения амперсанда при объявлении ссылок.
\begin{multicols}{3}
\noindent
\begin{lstlisting}[backgroundcolor = \color{correctcolor}]
int a = 100;
int* p = &a;
int& r = a;
\end{lstlisting}
\begin{lstlisting}[backgroundcolor = \color{wrongcolor}]
int a = 100;
int *p = &a;
int &r = a;
\end{lstlisting}
\begin{lstlisting}[backgroundcolor = \color{wrongcolor}]
int a = 100;
int * p = &a;
int & r = a;
\end{lstlisting}
\end{multicols}



\section{Избегайте использование \texttt{using}-директив для полного включения пространств имён}

\begin{multicols}{3}
\noindent
\begin{lstlisting}[backgroundcolor = \color{correctcolor}]
#include <iostream>

int main()
{
    std::cout << "Hello" << std::endl;
}
\end{lstlisting}
\begin{lstlisting}[backgroundcolor = \color{correctcolor}]
#include <iostream>
using std::cout, std::endl;

int main()
{
    cout << "Hello" << endl;
}
\end{lstlisting}
\begin{lstlisting}[backgroundcolor = \color{wrongcolor}]
#include <iostream>
using namespace std;

int main()
{
    cout << "Hello" << endl;
}
\end{lstlisting}
\end{multicols}
В заголовочных файлах вообще не стоит использовать никакие \texttt{using}-директивы.


\section{Стиль наименования указателей}
Имя указатель желательно начинать с символа \texttt{p}. Это относится как к обычным, так и к умным указателям.
\begin{multicols}{2}
\noindent
\begin{lstlisting}[backgroundcolor = \color{correctcolor}]
int number = 10;
int* pNumber = &number;
auto pData = std::make_unique(20);
\end{lstlisting}
\vfill
\begin{lstlisting}[backgroundcolor = \color{wrongcolor}]
int number = 10;
int* x = &number;
auto y = std::make_unique(20);
\end{lstlisting}
\end{multicols}

\newpage
\section{Стиль наименования приватных полей класса}
Приватные поля класса всегда начинаются с символа \texttt{m}
\begin{multicols}{2}
\noindent
\begin{lstlisting}[backgroundcolor = \color{correctcolor}]
class String
{
public:
    // ...
private:
   char*  mpData;
   size_t mSize;
   size_t mCapacity;
};
\end{lstlisting}
\vfill
\begin{lstlisting}[backgroundcolor = \color{wrongcolor}]
class String
{
public:
    // ...
private:
   char*  pData;
   size_t size;
   size_t capacity;
};
\end{lstlisting}
\end{multicols}

\section{Отступы}
В качестве отступов используется 4 пробела, а не знак табуляции.

\newpage
\begin{lstlisting}[backgroundcolor = \color{correctcolor}]
bool isPrime(int n)
{
    if (n <= 1)
        return false;
 
    for (int i = 2; i * i <= n; i++)
    {
        if (n % i == 0)
            return false;
    }
    return true;
}
\end{lstlisting}




\end{document}