\documentclass{article}
\usepackage[english,russian]{babel}
\usepackage{textcomp}
\usepackage{geometry}
  \geometry{left=2cm}
  \geometry{right=1.5cm}
  \geometry{top=1.5cm}
  \geometry{bottom=2cm}
\usepackage{tikz}
\usepackage{listings}
\usepackage{fancyvrb}
\usepackage{xcolor}
\usepackage{multicol}

\begin{document}
\pagenumbering{gobble}

\lstset{
  language=C++,
  basicstyle=\linespread{1.1}\ttfamily,
  columns=fixed,
  fontadjust=true,
  basewidth=0.5em,
  keywordstyle=\color{blue}\bfseries,
  commentstyle=\color{gray},
  texcl=true,
  stringstyle=\ttfamily\color{orange!50!black},
  showstringspaces=false,
  %numbers=false,
  numbersep=5pt,
  numberstyle=\tiny\color{black},
  numberfirstline=true,
  stepnumber=1,      
  numbersep=10pt,
  backgroundcolor=\color{white},
  showstringspaces=false,
  captionpos=b,
  breaklines=true
  breakatwhitespace=true,
  xleftmargin=.2in,
  extendedchars=\true,
  keepspaces = true,
  tabsize=4,
  upquote=true,
}
\lstset{ literate={~}{{\raisebox{0.5ex}{\texttildelow}}}{1}}

\renewcommand{\thesubsection}{\arabic{subsection}}
\makeatletter
\def\@seccntformat#1{\@ifundefined{#1@cntformat}%
   {\csname the#1\endcsname\quad}%    default
   {\csname #1@cntformat\endcsname}}% enable individual control
\newcommand\section@cntformat{}     % section level 
\newcommand\subsection@cntformat{Задача \thesubsection.\space} % subsection level
\newcommand\subsubsection@cntformat{\thesubsubsection.\space} % subsubsection level
\makeatother

\title{Семинар \#3: Инициализация. Домашнее задание.\vspace{-5ex}}\date{}\maketitle



\subsection{Виды инициализации}
Определите каким из видов инициализации (default, value, direct или copy) была инициализирована переменная в следующих строках. 
\begin{enumerate}
\item \lstinline|int a = 10;|
\item \lstinline|int a{};|
\item \lstinline|int a{10};|
\item \lstinline|int a;|
\item \lstinline|int a = {10};|
\item \lstinline|int a(10);|
\end{enumerate}
Для того, чтобы сдать эту задачу нужно создать файл в формате \texttt{.txt} и, используя любой текстовый редактор, записать в него ответы в следующем формате (ответы ниже неверны):
\begin{verbatim}
1) Default
2) Direct
3) Copy
\end{verbatim} 
После этого, файл нужно поместить в ваш репозиторий на github.


\subsection{Печать методов}
Пусть есть следующий класс:
\begin{lstlisting}
class Cat
{
	int x;
public:
	explicit Cat(int x) : x(x)   {std::cout << "Constructor from int\n";}
	Cat() : x(0)                 {std::cout << "Default Constructor\n";}
	Cat(const Cat& c) : x(c.x)   {std::cout << "Copy Constructor\n";}
	Cat& operator=(const Cat& c) {x = c.x; std::cout << "Assignment\n"; return *this;}
	~Cat()                       {std::cout << "Destructor\n";}
};
\end{lstlisting}
Определите, скомпилируется ли следующий код, использующий этот класс, и, если скомпилируется, то что будет напечатано в следующих программах:

\begin{enumerate}
\item \begin{Verbatim}[commandchars=\\\{\}]
\textcolor{blue}{int} main()
\{
    Cat a;
\}
\end{Verbatim}


\item \begin{Verbatim}[commandchars=\\\{\}]
\textcolor{blue}{int} main()
\{
    Cat a = 10;
\}
\end{Verbatim}

\item \begin{Verbatim}[commandchars=\\\{\}]
\textcolor{blue}{int} main()
\{
    Cat a\{10\};
\}
\end{Verbatim}

\item \begin{Verbatim}[commandchars=\\\{\}]
\textcolor{blue}{int} main()
\{
    Cat a\{\};
    Cat b\{a\};
\}
\end{Verbatim}

\item \begin{Verbatim}[commandchars=\\\{\}]
\textcolor{blue}{int} main()
\{
    Cat a;
    Cat b = a;
\}
\end{Verbatim}

\item \begin{Verbatim}[commandchars=\\\{\}]
\textcolor{blue}{int} main()
\{
    Cat a;
    Cat b;
    b = a;
\}
\end{Verbatim}

\item \begin{Verbatim}[commandchars=\\\{\}]
\textcolor{blue}{void} func(Cat a) \{\}
\textcolor{blue}{int} main()
\{
    Cat b;
    func(b);
\}
\end{Verbatim}

\item \begin{Verbatim}[commandchars=\\\{\}]
\textcolor{blue}{void} func(Cat\& a) \{\}
\textcolor{blue}{int} main()
\{
    Cat b;
    func(b);
\}
\end{Verbatim}

\item \begin{Verbatim}[commandchars=\\\{\}]
\textcolor{blue}{void} func(Cat a) \{\}
\textcolor{blue}{int} main()
\{
    func(10);
\}
\end{Verbatim}

\item \begin{Verbatim}[commandchars=\\\{\}]
\textcolor{blue}{void} func(Cat a) \{\}
\textcolor{blue}{int} main()
\{
    func(Cat\{10\});
\}
\end{Verbatim}

\item \begin{Verbatim}[commandchars=\\\{\}]
\textcolor{blue}{class} Dog
\{
    Cat y;
\textcolor{blue}{public}:
    Dog(\textcolor{blue}{const} Cat& a) : y(a) \{\}
\};

\textcolor{blue}{int} main()
\{
    Cat a;
    Dog b(a);
\}
\end{Verbatim}

\item \begin{Verbatim}[commandchars=\\\{\}]
\textcolor{blue}{class} Dog
\{
    Cat y;
\textcolor{blue}{public}:
    Dog(\textcolor{blue}{const} Cat& a) \{y = a;\}
\};

\textcolor{blue}{int} main()
\{
    Cat a;
    Dog b(a);
\}
\end{Verbatim}

\item \begin{Verbatim}[commandchars=\\\{\}]
\textcolor{blue}{int} main()
\{
    Cat* p = \textcolor{blue}{new} Cat;
    \textcolor{blue}{delete} p;
\}
\end{Verbatim}

\item \begin{Verbatim}[commandchars=\\\{\}]
\textcolor{blue}{#include} <cstdlib>
\textcolor{blue}{int} main()
\{
    Cat* p = (Cat*)std::malloc(3 * \textcolor{blue}{sizeof}(Cat));
    std::free(p);
\}
\end{Verbatim}

\item \begin{Verbatim}[commandchars=\\\{\}]
\textcolor{blue}{int} main()
\{
    Cat* p = \textcolor{blue}{new} Cat[3];
    \textcolor{blue}{delete}[] p;
\}
\end{Verbatim}

\item \begin{Verbatim}[commandchars=\\\{\}]
\textcolor{blue}{#include} <vector>
\textcolor{blue}{int} main()
\{
    std::vector<Cat> v(3);
\}
\end{Verbatim}

\item \begin{Verbatim}[commandchars=\\\{\}]
\textcolor{blue}{struct} Dog
\{
    \textcolor{blue}{static} Cat y;
\};
Cat Dog::y\{\};

\textcolor{blue}{int} main()
\{
    Dog a;
    Dog b;
\}
\end{Verbatim}



\end{enumerate}
Для того, чтобы сдать эту задачу нужно создать файл в формате \texttt{.txt} и, используя любой текстовый редактор, записать в него ответы в следующем формате (ответы ниже неверны):
\begin{verbatim}
1) Copy Constructor
   Destructor
2) Constructor from int
   Assignment
   Destructor
3) Error
\end{verbatim} 
После этого, файл нужно поместить в ваш репозиторий на github.



\subsection{\texttt{new}}
Используйте операторы \texttt{new} или \texttt{new[]}, чтобы создать в куче и сразу инициализировать следующие объекты:
\begin{itemize}
\item Один объект типа \texttt{int}, равный \texttt{123}.
\item Один объект типа \texttt{std::string}, равный \texttt{"Cats and Dogs"}.
\item Маcсив объектов типа \texttt{int}, содержащий элементы \texttt{10, 20, 30, 40, 50}.
\item Один объект типа \texttt{std::vector}, содержащий элементы \texttt{10, 20, 30, 40, 50}.
\item Маcсив объектов типа \texttt{std::string}, равный \texttt{\{"Cat"{}, "Dog"{}, "Mouse"\}}.
\end{itemize}
Все эти объекты обязательно должны быть созданы в куче, а не на стеке.
Напечатайте все созданные объекты на экран.
Удалите все созданные объекты с помощью операторов \texttt{delete} и \texttt{delete[]}.


\subsection{Ссылка как поле}
Есть некоторое количество групп кошек, которые соревнуются в том, кто поймает больше мышек. Соревнование командное, побеждает та группа, которая суммарно поймала больше мышек.

Ваша задача - написать класс \texttt{Cat}, который оказался бы полезным для автоматизации подсчёта пойманных мышек. Сумарное количество пойманых мышек для каждой группы будет хранится вне класса \texttt{Cat}, например в локальной переменной (так как кошки из других групп не должны иметь доступ к этой переменной). В самом же классе должна храниться \textit{ссылка} на эту переменную. Хоть задачу можно решить и с помощью указателя,  но в данной задаче обязательно использовать ссылку. В классе вам нужно написать следующие методы:

\begin{itemize}
\item Конструктор. Должен конструироваться от переменной, в которой будет храниться суммарное количество мышек.
\item Метод \texttt{catchMice} - поймать мышек. Принимает число и увеличивает суммарно количество пойманых мышек данной группы на это число.
\end{itemize}

Пример использования такого класса (код ниже должен работать и с вашим классом):

\begin{lstlisting}
int miceCaughtA = 0;
int miceCaughtB = 0;

Cat alice(miceCaughtA), alex(miceCaughtA), anna(miceCaughtA);
Cat bob(miceCaughtB), bella(miceCaughtB);

alice.catchMice(2);
alex.catchMice(1);
bella.catchMice(4);
bob.catchMice(2);
anna.catchMice(1);
bella.catchMice(1);
alex.catchMice(4);
bella.catchMice(5);
alice.catchMice(2);

cout << miceCaughtA << endl; // Должно напечатать 10
cout << miceCaughtB << endl; // Должно напечатать 12
\end{lstlisting}


\subsection{Сумма из строки}
Напишите функцию, которая принимает на вход строку в следующем формате: \texttt{"[num1, num2, ... numN]"}.
Функция должна возвращать целое число типа \texttt{int} -- сумму всех чисел в квадратных скобках. В случае, если на вход приходит некорректная строка, то функция должна бросать исключение \texttt{std::invalid\_argument}. Протестируйте эту функцию в \texttt{main}, обработав бросаемое исключение.
\begin{center}
\begin{tabular}{ l | l }
 аргумент & возвращаемое значение \\ \hline
 \texttt{"[10, 20, 30, 40, 50]"} & \texttt{150} \\
 \texttt{"[4, 8, 15, 16, 23, 42]"} & \texttt{108}  \\ 
 \texttt{"[20]"} & \texttt{20} \\
 \texttt{"[]"} & \texttt{0} \\
\end{tabular}
\end{center}








\newpage
\section{Необязательные задачи}


\subsection{Создание \texttt{mipt::String}}
В файле \texttt{miptstring.hpp} содержится класс \texttt{mipt::String}. Создайте объекты этого класса в следующим образом:
\begin{itemize}
\item Создайте объект класса \texttt{mipt::String} на стеке обычным образом и инициализируйте его строкой \texttt{Cat}.
\item Создайте объект класса \texttt{mipt::String} в куче, используя оператор \texttt{new}, и инициализируйте его строкой \texttt{Dog}. Напечатайте этот объект на экран. Удалите этот объект, с помощью оператора \texttt{delete}.
\item Пусть у нас есть массив:
\begin{lstlisting}
char x[sizeof(mipt::String)];
\end{lstlisting}
Создайте объект класса \texttt{mipt::String} в этом массиве с помощью оператора \texttt{placement new} и инициализируйте объект строкой \texttt{Elephant}. Напечатайте этот объект на экран. Удалите этот объект.
\end{itemize}


\subsection{\texttt{StringView}}
Создайте свой класс \texttt{mipt::StringView}, аналог класса \texttt{std::string\_view} для класса \texttt{mipt::String}.
Этот класс должен содержать 2 поля указатель \texttt{mpData} (тип \texttt{const char*}) и размер \texttt{mSize} (тип \texttt{size\_t}). Класс \texttt{mipt::String} можно найти в файле \texttt{miptstring.hpp}.\\

Методы, которые нужно реализовать:
\begin{itemize}
\item Конструктор по умолчанию. Должен устанавливать указатель в \texttt{nullptr}, а размер в 0.
\item Конструктор копирования.
\item Конструктор от \texttt{mipt::String}.
\item Конструктор от \texttt{const char*}
\item Перегруженный \texttt{operator[]}
\item Метод \texttt{at}, аналог \texttt{operator[]}, но если индекс выходит за границы, то данный метод должен бросать исключение \texttt{std::out\_of\_range}
\item Перегруженный \texttt{operator<}
\item Перегруженный \texttt{operator<{}<} с объектом \texttt{std::ostream}.
\item Метод \texttt{size}.
\item Метод \texttt{substr}, должен возвращать объект типа \texttt{std::string\_view}.
\item Методы \texttt{remove\_prefix} и \texttt{remove\_suffix}.
\end{itemize}

Также придётся изменить класс \texttt{mipt::String}. Нужно будет добавить конструктор от \texttt{mipt::StringView}.
Протестируйте этот класс.

\end{document}