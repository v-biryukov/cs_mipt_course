\documentclass{article}
\usepackage[utf8x]{inputenc}
\usepackage{ucs}
\usepackage{amsmath} 
\usepackage{amsfonts}
\usepackage{marvosym}
\usepackage{wasysym}
\usepackage{upgreek}
\usepackage[english,russian]{babel}
\usepackage{graphicx}
\usepackage{float}
\usepackage{textcomp}
\usepackage{hyperref}
\usepackage{geometry}
  \geometry{left=2cm}
  \geometry{right=1.5cm}
  \geometry{top=1cm}
  \geometry{bottom=2cm}
\usepackage{tikz}
\usepackage{ccaption}
\usepackage{multicol}

\hypersetup{
   colorlinks=true,
   citecolor=blue,
   linkcolor=black,
   urlcolor=blue
}

\usepackage{listings}
%\setlength{\columnsep}{1.5cm}
%\setlength{\columnseprule}{0.2pt}

\usepackage[absolute]{textpos}


\usepackage{colortbl,graphicx,tikz}
\definecolor{X}{rgb}{.5,.5,.5}

\renewcommand{\thesubsection}{\arabic{subsection}}

\begin{document}
\pagenumbering{gobble}
\lstset{
  language=C++,                % choose the language of the code
  basicstyle=\linespread{1.1}\ttfamily,
  columns=fixed,
  fontadjust=true,
  basewidth=0.5em,
  keywordstyle=\color{blue}\bfseries,
  commentstyle=\color{gray},
  stringstyle=\ttfamily\color{orange!50!black},
  showstringspaces=false,
  numbersep=5pt,
  numberstyle=\tiny\color{black},
  numberfirstline=true,
  stepnumber=1,                   % the step between two line-numbers.        
  numbersep=10pt,                  % how far the line-numbers are from the code
  backgroundcolor=\color{white},  % choose the background color. You must add \usepackage{color}
  showstringspaces=false,         % underline spaces within strings
  captionpos=b,                   % sets the caption-position to bottom
  breaklines=true,                % sets automatic line breaking
  breakatwhitespace=true,         % sets if automatic breaks should only happen at whitespace
  xleftmargin=.2in,
  extendedchars=\true,
  keepspaces = true,
}
\lstset{literate=%
   *{0}{{{\color{red!20!violet}0}}}1
    {1}{{{\color{red!20!violet}1}}}1
    {2}{{{\color{red!20!violet}2}}}1
    {3}{{{\color{red!20!violet}3}}}1
    {4}{{{\color{red!20!violet}4}}}1
    {5}{{{\color{red!20!violet}5}}}1
    {6}{{{\color{red!20!violet}6}}}1
    {7}{{{\color{red!20!violet}7}}}1
    {8}{{{\color{red!20!violet}8}}}1
    {9}{{{\color{red!20!violet}9}}}1
}
\newcommand\upquote[1]{\textquotesingle#1\textquotesingle}

\renewcommand{\thesubsection}{\arabic{subsection}}
\makeatletter
\def\@seccntformat#1{\@ifundefined{#1@cntformat}%
   {\csname the#1\endcsname\quad}%    default
   {\csname #1@cntformat\endcsname}}% enable individual control
\newcommand\section@cntformat{}     % section level 
\newcommand\subsection@cntformat{Задача \thesubsection.\space} % subsection level
\newcommand\subsubsection@cntformat{\thesubsubsection.\space} % subsubsection level
\makeatother


\makeatletter
\newcount\my@repeat@count
\newcommand{\myrepeat}[2]{%
  \begingroup
  \my@repeat@count=\z@
  \@whilenum\my@repeat@count<#1\do{#2\advance\my@repeat@count\@ne}%
  \endgroup
}
\makeatother

\title{Семинар \#6: Итераторы и алгоритмы. Домашнее задание.\vspace{-5ex}}\date{}\maketitle
В задачах (\texttt{1} - \texttt{5} включительно) нельзя использовать циклы. Их нужно решить, используя алгоритмы STL.
\subsection{Горка}
На вход программе подаётся $n$ чисел. Найдите первый максимум среди этих чисел. Отсортируйте часть массива, которая идёт до этого максимума по возрастанию. А часть массива, которая идёт после первого максимума отсортируйте по убыванию.
\begin{center}
\begin{tabular}{ l | l }
 вход & выход \\ \hline
 \texttt{10} & \texttt{1 2 5 5 6 8 8 6 4 3} \\
 \texttt{5 2 1 5 6 8 6 4 3 8} &
\end{tabular}
\end{center}


\subsection{Обращение вектора строк}
Напишите функцию, которая принимает на вход вектор строк и обращает сам вектор, а также каждую его строку.
\begin{center}
\begin{tabular}{ l | l }
 аргумент & выход \\ \hline
 \texttt{\{"cat"{}, "dog"{}, "mouse"{}, "elephant"\}} & \texttt{\{"tnahpele"{}, "esuom"{}, "god"{}, "tac"\}} \\
 \texttt{\{"a"{}, "bc"\}} & \texttt{\{"cb"{}, "a"\}} \\
\end{tabular}
\end{center}

\subsection{Проверка на верхний регистр}
Напишите функцию, которая будет принимать на вход строку и проверять находится ли эта строка в верхнем регистре.
\begin{center}
\begin{tabular}{ l | l }
 аргумент & выход \\ \hline
 \texttt{"Cats and Dogs!"} & \texttt{false} \\
 \texttt{"CATS AND DOGS!"} & \texttt{true} \\
 \texttt{"ABc123!\#?"} & \texttt{false} \\
 \texttt{"ABC123!\#?"} & \texttt{true}
\end{tabular}
\end{center}

\subsection{Идентификатор}
Напишите функцию \texttt{bool isIdentifier(std::string\_view s)}, которая будет принимать на вход строку и проверяет является ли эта строка допустимым идентификатором в языке C++. Случаи, когда приходящая на вход строка является одним из ключевых слов языка C++, можно не рассматривать. 
\begin{center}
\begin{tabular}{ l | l }
 аргумент & выход \\ \hline
 \texttt{"a"} & \texttt{true} \\
 \texttt{"isIdentifier"} & \texttt{true} \\
 \texttt{"\_name123"} & \texttt{true} \\
 \texttt{"hello world"} & \texttt{false} \\
 \texttt{"123name"} & \texttt{false} \\
 \texttt{"my-name"} & \texttt{false} \\
 \texttt{"int"} & \texttt{true}\\
\end{tabular}
\end{center}


\subsection{Передвинуть пробелы}
Напишите функцию, которая будет принимать на вход строку по ссылке и передвигать все её пробелы в конец.
\begin{center}
\begin{tabular}{ l | l }
 аргумент & выход \\ \hline
 \texttt{"cats and dogs"} & \texttt{"catsanddogs \quad "} \\
 \texttt{"   cats and  dogs"} & \texttt{"catsanddogs  \quad\quad\quad   "} \\
\end{tabular}
\end{center}



\subsection{Шаблонный максимум}
Напишите шаблонную функцию \texttt{maxElement}, которая должна будет принимать 2 итератора и возвращать максимальный элемент на диапазоне, задаваемом этими итераторами (как лучше вернуть элемент? по ссылке или по значению?). Элементы контейнера должны сравнимы с помощью оператора меньше (\texttt{<}). Протестируйте эту функцию на различных контейнерах (\texttt{std::vector}, \texttt{std::list}, \texttt{std::set}).

\subsection{Шаблонный обмен соседних}
Напишите шаблонную функцию \texttt{swapNeighbours}, которая должна будет принимать 2 итератора и менять местами пары соседних элементов на диапазоне, задаваемом входящими итераторами. Если в диапазоне нечётное количество элементов, то последний элемент должен остаться на месте. То есть, если диапазон содержал элементы \texttt{\{10, 20, 30, 40, 50\}}, то после исполнения этой функции элементы диапазона должны иметь вид \texttt{\{20, 10, 40, 30, 50\}} Для обмена элементов используйте функцию \texttt{std::swap}. Протестируйте эту функцию на различных контейнерах (\texttt{std::vector}, \texttt{std::list}, \texttt{std::forward\_list}).


\subsection{Поиск соседей}
Напишите шаблонную функцию \texttt{bestNeighbours}, которая должна будет принимать 2 итератора и функциональный объект, принимающий 2 элемента. Функция должна находить такую пару соседних элементов, что результат применения функционального объекта к этой паре будет наибольшим. Функция должна возвращать итератор на первый элемент этой пары. Например, если у нас есть такой вектор:
\begin{lstlisting}
std::vector<int> v {50, 10, 10, 20, 90, 30, 40, 60, 80, 20};
\end{lstlisting}
То если мы применим к нему функцию  \texttt{bestNeighbours} вот так:
\begin{lstlisting}
auto it = bestNeighbours(v.begin(), v.end(), [](int a, int b){return a + b;});
\end{lstlisting}
то функция должна вернуть итератор на восьмой элемент (60), потому что пара элементов 60 и 80 имеют наибольшую сумму.
А если мы применим эту функцию вот так: 
\begin{lstlisting}
auto it = bestNeighbours(v.begin(), v.end(), [](int a, int b){return std::abs(a - b);});
\end{lstlisting}
то функция должна вернуть итератор на четвёртый элемент (20), потому что пара элементов 20 и 90 имеют наибольший модуль разности.
Протестируйте эту функцию на различных контейнерах (\texttt{std::vector}, \texttt{std::list}, \texttt{std::set}, \texttt{std::forward\_list}).


\end{document}
