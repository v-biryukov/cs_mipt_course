\documentclass{article}
\usepackage[english,russian]{babel}
\usepackage{textcomp}
\usepackage{geometry}
  \geometry{left=2cm}
  \geometry{right=1.5cm}
  \geometry{top=1.5cm}
  \geometry{bottom=2cm}
\usepackage{tikz}
\usepackage{listings}
\usepackage{fancyvrb}
\usepackage{xcolor}
\pagenumbering{gobble}

\lstset{
  language=C++,
  basicstyle=\linespread{1.1}\ttfamily,
  columns=fixed,
  fontadjust=true,
  basewidth=0.5em,
  keywordstyle=\color{blue}\bfseries,
  commentstyle=\color{gray},
  texcl=true,
  stringstyle=\ttfamily\color{orange!50!black},
  showstringspaces=false,
  %numbers=false,
  numbersep=5pt,
  numberstyle=\tiny\color{black},
  numberfirstline=true,
  stepnumber=1,      
  numbersep=10pt,
  backgroundcolor=\color{white},
  showstringspaces=false,
  captionpos=b,
  breaklines=true
  breakatwhitespace=true,
  xleftmargin=.2in,
  extendedchars=\true,
  keepspaces = true,
  tabsize=4,
  upquote=true,
}
\lstset{ literate={~}{{\raisebox{0.5ex}{\texttildelow}}}{1}}

\renewcommand{\thesubsection}{\arabic{subsection}}
\makeatletter
\def\@seccntformat#1{\@ifundefined{#1@cntformat}%
   {\csname the#1\endcsname\quad}%    default
   {\csname #1@cntformat\endcsname}}% enable individual control
\newcommand\section@cntformat{}     % section level 
\newcommand\subsection@cntformat{Задача \thesubsection.\space} % subsection level
\newcommand\subsubsection@cntformat{\thesubsubsection.\space} % subsubsection level
\makeatother

\newcommand\upquote[1]{\textquotesingle#1\textquotesingle}

\begin{document}
\title{Семинар \#4: Шаблоны. Домашнее задание.\vspace{-5ex}}\date{}\maketitle




\subsection{Шаблонный куб}
Напишите шаблонную функцию \texttt{cube} которая будет принимать число любого типа и возвращать куб этого числа того же типа. Пример работы с такой функцией:
\begin{lstlisting}
auto a = cube(5);
std::cout << a << " " << sizeof(a) << std::endl;  // Должен напечатать  125 4

auto b = cube(5.0);
std::cout << b << " " << sizeof(b) << std::endl;  // Должен напечатать  125.0 8

char x = 5;
auto c = cube(x);
std::cout << c << " " << sizeof(c) << std::endl;  // Должен напечатать  125 1
\end{lstlisting}


\subsection{Утроение}
Пусть была написана шаблонная функция \texttt{triple}, которая может принять на вход число любого типа и вернуть это число, умноженное на 3. Функция работает с числами, но нам бы хотелось чтобы эта же шаблонная функция работала и со строками типа \texttt{std::string}. Измените код программы, так чтобы можно было использовать функцию \texttt{triple} для объектов типа \texttt{std::string}.
\begin{lstlisting}
#include <iostream>

template<typename T>
T triple(const T& x)
{
	return 3 * x;
}

int main()
{
	int a = 10;
	std::cout << triple(a) << std::endl;  // Сработает, напечатает 30
	
	std::string b = "Cat";
	std::cout << triple(b) << std::endl;  // Ошибка, нельзя число 3 умножать на std::string
	                                      // Нужно чтобы напечаталось CatCatCat
}
\end{lstlisting}
Решите эту задачу тремя способами:
\begin{enumerate}
\item Измените шаблонную функцию \texttt{triple}, так чтобы она работала со строками тоже (вместо умножения используйте сложение).
\item Используйте перегрузку операторов и напишите \lstinline|std::string operator*(int n, std::string s)|. После этого строки \texttt{std::string} могут быть использованы в шаблонной функции \texttt{triple}.
\item Используйте перегрузку функций и напишите отдельную перегрузку для функции \texttt{triple}, принимающую объект типа \texttt{std::string}.
 
\end{enumerate}


\subsection{Минимум и максимум в векторе}
Напишите шаблонную функцию, которая будет принимать на вход вектор элементов некоторого типа и возвращать пару, содержащую минимальный и максимальный элемент этого вектора. Известно, что вектор содержит хотя бы один элемент.
\begin{lstlisting}
template<typename T>
std::pair<T,T> minmax(const std::vector<T>& v)
\end{lstlisting}
Протестируйте данную функцию на векторах, содержащих объекты следующих типов: \texttt{int}, \texttt{std::string} и \texttt{std::pair<int, int>}:
\begin{lstlisting}
int main()
{
	std::vector<int> a {60, 10, 40, 80, 30};
	auto am = minmax(a);
	std::cout << am.first << " " << am.second << std::endl;  // 10 80
	
	std::vector<std::string> b {"Cat", "Dog", "Mouse", "Camel", "Wolf"};
	auto bm = minmax(b);
	std::cout << bm.first << " " << bm.second << std::endl;  // Camel Wolf

	std::vector<std::pair<int, int>> c {{10, 90}, {30, 10}, {20, 40}, {10, 50}};
	auto cm = minmax(c);
	std::cout << cm.first.first << " " << cm.first.second << std::endl;    // 10 50
	std::cout << cm.second.first << " " << cm.second.second << std::endl;  // 30 10
}
\end{lstlisting}



\subsection{Сравнение количества элементов в контейнере}
Напишите шаблонную функцию \texttt{hasMoreElements}, которая бы принимала на вход два контейнера и возвращала бы \texttt{true}, если количество элементов в первом контейнере болеше чем во втором и \texttt{false} иначе. Под контейнером тут понимается \texttt{std::vector}, \texttt{std::array}, \texttt{std::string} или любой другой класс, хранящий набор элементов и имеющий метод \texttt{size}.
\begin{lstlisting}
int main()
{
	std::vector<int> a {10, 20, 30, 40, 50};
	std::string b = "Cat";
	std::string c = "Elephant";
	std::array<int, 3> d {10, 20, 30};
	
	std::cout << hasMoreElements(a, b) << std::endl;  // Должно напечатать 1
	std::cout << hasMoreElements(a, c) << std::endl;  // Должно напечатать 0
	std::cout << hasMoreElements(a, d) << std::endl;  // Должно напечатать 1
}
\end{lstlisting}


\newpage
\subsection{Изменение порядка байт}
Напишите шаблонную функцию \texttt{swapEndianness}, которая бы меняла порядок байт объекта скалярного типа с Little Endian на Big Endian или наоборот.
\begin{lstlisting}
int main()
{
    std::cout << std::hex;
    
    int a = 0x1a2b3c4d; 
    std::cout << a << std::endl;  // Должен напечатать 1a2b3c4d
    swapEndianness(a);             
    std::cout << a << std::endl;  // Должен напечатать 4d3c2b1a
    swapEndianness(a);             
    std::cout << a << std::endl;  // Должен напечатать 1a2b3c4d
    
    short b = 0x1a2b;   
    std::cout << b << std::endl;  // Должен напечатать 1a2b
    swapEndianness(b);
    std::cout << b << std::endl;  // Должен напечатать 2b1a
}
\end{lstlisting}


\subsection{Целые числа для вычисления по модулю}
Напишите шаблонный класс \texttt{Modular}, который будет представлять собой целые числа с модульной арифметикой. У класса должно быть 2 шаблонных параметра: тип целого числа, который будет использоваться для хранения модульного числа и сам модуль. Напишите следующие методы:
\begin{enumerate}
\item Конструктор от целого числа.
\item Конструктор копирования.
\item Операторор присваивания от такого же типа.
\item Перегруженные бинарные операторы сложения, вычитания, умножения с числами и с объектами такого же типа.
\item Унарный оператор минус.
\item Оператор \texttt{<{}<} с объектом \texttt{std::ostream} для вывода на экран.
\item Конструктор от типа \texttt{Modular} с другими шаблонными параметрами.
\end{enumerate}
\begin{lstlisting}
Modular<int, 7> a(10);
std::cout << a << std::endl; // Напечатает 3
a = (a + 8) * 4;
std::cout << a << std::endl; // Напечатает 2

Modular<int, 7> b(a);
b = b + 2;
a = a - b;
std::cout << a << std::endl; // Напечатает 5

Modular<short, 3> с(a);
std::cout << c << std::endl; // Напечатает 2
\end{lstlisting}


\newpage
\subsection{Менеджер создания объекта}
Напишите шаблонный класс \texttt{Manager}, который будет разделять процессы выделения/освобождения памяти и создания/уничтожения объекта. У этого класса должны быть следующие методы:
\begin{itemize}
\item Конструктор по умолчанию.
\item \texttt{allocate()} - будет выделять необходимое количество памяти под объект типа \texttt{T} в куче (используйте \texttt{std::malloc}).
\item \texttt{construct(const T\& t)} - будет создавать объект типа \texttt{T}, используя конструктор копирования, в выделенной памяти. Используйте оператор \texttt{placement new}.
\item \texttt{destruct()} - будет уничтожать объект в выделенной памяти.
\item \texttt{dealocate()} - будет освобождать выделенную памяти.
\item \texttt{get} - будет возвращать ссылку на объект.
\end{itemize}
Пример использования данного класса:
\begin{lstlisting}
Manager<std::string> a;
a.allocate();

a.construct("Cats and dogs");
a.get() += " and elephant";
cout << a.get() << endl;  // Должен напечатать Cats and dogs and elephant
a.destruct();

a.construct("Sapere Aude");
cout << a.get() << endl;  // Должен напечатать Sapere Aude
a.destruct();

a.deallocate();
\end{lstlisting}


\newpage



\newpage
\section*{Необязательные задачи (не входят в ДЗ, никак не учитываются)}
\setcounter{subsection}{0}

\subsection{Простые делители}
Напишите функцию \texttt{std::vector<std::pair<int, int>{}> factorization(int n)}, которая будет находить все простые делители числа \texttt{n} и их количества.
\begin{center}
\begin{tabular}{ l | l }
 аргумент & возвращаемое значение \\ \hline
 \texttt{60} & \texttt{\{\{2, 2\}, \{3, 1\}, \{5, 1\}\}} \\
 \texttt{626215995} & \texttt{\{\{3, 3\}, \{5, 1\}, \{17, 1\}, \{29, 1\}, \{97, 2\}\}} \\
 \texttt{107} & \texttt{\{\{107, 1\}\}} \\
 \texttt{1} & \texttt{\{\{1, 1\}\}} 
\end{tabular}
\end{center}

\subsection{Времена из строки}
\begin{itemize}
\item Напишите простой класс для работы со временем:
\begin{lstlisting}
class Time 
{
private:
    int mHours, mMinutes, mSeconds;
public:
    Time(int hours, int minutes, int seconds);
    Time(const std::string& s); // строка в формате "hh:mm:ss"
    Time operator+(Time b) const;
    int hours() const; int minutes() const; int seconds() const;
    friend std::operator<<(std::ostream& out, Time t);
};
\end{lstlisting}
\item Напишите функцию \texttt{std::vector<Time> getTimesFromString(const std::string\& s)}, которая будет принимать строку в формате \texttt{"hh:mm:ss ... hh:mm:ss"}, где за место букв должны стоять некоторые числа. Например, строка может иметь вид \texttt{"11:20:05 05:45:30 22:10:45"}. Функция должна возвращать вектор времен, соответствующий временам в строке.
\item Напишите функцию \texttt{Time sumTimes(const std::vector<Time>\& v)}, которая будет суммировать все времена и возвращать эту сумму. Для суммирования времён используйте перегруженный оператор \texttt{+} класса \texttt{Time}.
\end{itemize}
Использовать функции можно следующим образом:
\begin{lstlisting}
std::vector<Time> v = getTimesFromString("11:20:05 05:45:30 22:10:45");
v.push_back(Time("01:10:30"));
Time s = sumTimes(v);
std::cout << s << std::endl;
\end{lstlisting}
В результате исполнения этого участка кода на экран должно напечататься \texttt{16:26:50}.



\subsection{Указатель или ссылка?}
Напишите шаблонный класс \texttt{Ref<T>} который будет совмещать свойства указателя и ссылки. Как и указатель этот объект можно будет копировать и ложить в контейнеры. Но инициалироваться данный объект должен как ссылка и все операторы, применяемые к этому объекту, должны применяться к тому объекту, на который он ссылается. Правда, к сожалению, перегрузить оператор точка (пока?) нельзя, поэтому вместо этого будем использовать оператор \texttt{->}.

\begin{itemize}
\item Конструктор от объекта типа \texttt{T}.
\item Конструктор копирования.
\item Оператор присваивания. Присваивание должно производится к объекту, на который ссылается \texttt{Ref}.
\item Оператор \texttt{+=}.
\item Оператор \texttt{+}. Должен возвращать новый объект типа \texttt{T}.
\item Перегруженный оператор \texttt{->}
\item Функция \texttt{get}. Должна возвращать ссылку на объект, на который \texttt{Ref} ссылается.
\item Дружественный оператор \texttt{operator<{}<(std::ostream\&, Ref<T>)}. 
\end{itemize}

Код для тестирования:
\begin{lstlisting}
void toUpper(Ref<std::string> r)
{
    for (size_t i = 0; i < r->size(); ++i)
        r.get()[i] = toupper(r.get()[i]);
}
int main()
{
    int a = 10;
    Ref<int> ra = a;
    ra += 10;
    cout << a << " " << ra << endl;

    std::string s = "Cat";
    Ref<std::string> rs = s;
    rs = "Mouse";
    rs += "Elephant";
    cout << rs << endl;
    cout << s  << endl;
    
    toUpper(s);
    cout << s << endl;

    std::vector<std::string> animals {"Cat", "Dog", "Elephant", "Worm"};
    std::vector<Ref<std::string>> refs {animals.begin(), animals.end()};

    for (int i = 0; i < refs.size(); ++i)
        refs[i] += "s";

    for (int i = 0; i < animals.size(); ++i)
        cout << animals[i] << " ";
    cout << endl;
}
\end{lstlisting}
Этот код должен напечатать:
\begin{verbatim}
20 20
MouseElephant
MouseElephant
MOUSEELEPHANT
Cats Dogs Elephants Worms 
\end{verbatim}


\end{document}
