\documentclass{article}
\usepackage[english,russian]{babel}
\usepackage{textcomp}
\usepackage{geometry}
  \geometry{left=2cm}
  \geometry{right=1.5cm}
  \geometry{top=1.5cm}
  \geometry{bottom=2cm}
\usepackage{tikz}
\usepackage{multicol}
\usepackage{listings}
\pagenumbering{gobble}

\lstdefinestyle{csMiptCppStyle}{
  language=C++,
  basicstyle=\linespread{1.1}\ttfamily,
  columns=fixed,
  fontadjust=true,
  basewidth=0.5em,
  keywordstyle=\color{blue}\bfseries,
  commentstyle=\color{gray},
  texcl=true,
  stringstyle=\ttfamily\color{orange!50!black},
  showstringspaces=false,
  %numbers=false,
  numbersep=5pt,
  numberstyle=\tiny\color{black},
  numberfirstline=true,
  stepnumber=1,      
  numbersep=10pt,
  backgroundcolor=\color{white},
  showstringspaces=false,
  captionpos=b,
  breaklines=true
  breakatwhitespace=true,
  xleftmargin=.2in,
  extendedchars=\true,
  keepspaces = true,
  tabsize=4,
  upquote=true,
}
\lstdefinestyle{csMiptCppBorderStyle}{
  style=csMiptCppStyle,
  framexleftmargin=5mm, 
  frame=shadowbox, 
  rulesepcolor=\color{gray}
}

\lstset{style=csMiptCppStyle}
\lstset{literate={~}{{\raisebox{0.5ex}{\texttildelow}}}{1}}


\begin{document}
\title{Семинар \#4: Шаблоны. \vspace{-5ex}}\date{}\maketitle

\textbf{Шаблоны} (англ. \textit{templates}) -- это механизм языка C++, который позволяет писать обобщённый код, то есть код, который может работать с различными типами данных.

\section*{Шаблоны функций}
\subsection*{Написание обобщённых функций}
Одна из проблем языков программирования, не поддерживающих шаблоны или аналогичные механизмы, заключается в необходимости создавать множество схожих функций для каждого типа данных. 

В одном из предыдущих семинаров уже рассматривалсь пример функции нахождения модуля числа. В языке C нет шаблонов, поэтому функцию нахождения модуля нужно было написать для каждого типа и дать каждой из этих функций уникальное имя (в стандартной библиотеке C эти функции назывались: \texttt{abs} для нахождения модуля числа типа \texttt{int}, \texttt{fabs} -- для чисел типа \texttt{double} и т. д.). В результате код на языке C выглядел так:
\begin{lstlisting}
int abs(int x)
{
	if (x < 0)
		return -x;
	return x;
}
double fabs(double x)
{
	if (x < 0)
		return -x;
	return x;
}
// И так далее ещё много раз для всех возможных типов
\end{lstlisting}
Этот код имеет следующие проблемы:
\begin{enumerate}
\item Придётся придумывать и запоминать разные странные названия функций для разных входных типов. Легко случайно ошибиться и вызвать функцию для другого типа, что может привести к ошибке.

\item Приходится писать много похожих функций. Получается очень много повторяемого кода.
\end{enumerate}
В языке C++ есть механизм перегрузки функций, который позволяет улучшить этот код, дав каждой функции одинаковое имя. Перегрузка функций решает первую проблему, но не решает вторую.
\begin{lstlisting}
int abs(int x)
{
	if (x < 0)
		return -x;
	return x;
}
double abs(double x)
{
	if (x < 0)
		return -x;
	return x;
}
// И так далее ещё много раз для всех возможных типов
\end{lstlisting}

\newpage
\noindent Вторую проблему можно решить, используя \textit{шаблон функции} (для этого понятия есть ещё один эквивалентный термин -- \textit{шаблонная функция}). Шаблоны создаются с помощью ключевого слова \texttt{template}, после которого в угловых скобках указываются параметры шаблона. Самым частым параметром шаблона является тип, такие параметры обозначаются ключевым словом \texttt{typename}. Шаблонная функция вычисления модуля числа будет выглядеть следующим образом:
\begin{lstlisting}
#include <iostream>

template<typename T>
T abs(T x)
{
	if (x < 0)
		return -x;
	return x;
}

int main()
{
	std::cout << abs<int>(-5) << std::endl;       // Напечатает 5
	std::cout << abs<double>(-1.5) << std::endl;  // Напечатает 1.5
}
\end{lstlisting}
Пояснения по этому коду:
\begin{itemize}
\item В данной программе написана шаблонная функция \texttt{abs}. Шаблонная функция не является обычной функцией. Это новая абстракция, которая не относится ни к функциям, ни к классам, ни к другим понятиям, с которыми мы сталкивались в этом курсе.

\item Шаблонная функция используется для генерации обычных функций. Например, если компилятор увидит в коде:
\begin{lstlisting}
abs<int>
\end{lstlisting}
то он сгенерирует новую обычную функцию из шаблонной функции \texttt{abs}. Для этого компилятор просто возьмёт шаблон функции и везде вмето \texttt{T} подставит \texttt{int}. Таким образом из шаблона \texttt{abs} получится обычная функция \texttt{abs<int>}. Процесс получение обычной функции из шаблона называется \textit{инстанцированием}. После того как компилятор получил обычную функцию из шаблонной, он будет её компилировать также как он компилирует другие обычные функции.

\item Параметры шаблона задаются в угловых скобках. В данном примере шаблонная функция \texttt{abs} имеет один шаблонный параметр под названием \texttt{T}, который имеет "тип"{} \texttt{typename}. У шаблонов может быть и несколько параметров разных типов.

\itemИспользование имени \texttt{T} для параметра шаблона является общепринятым соглашением. Но никто не запрещает использовать и другие имена. Однако имеется правило хорошего стиля, которое говорит о том, что параметры шаблонов следует именовать с заглавной буквы, например, так:
\begin{lstlisting}
template<typename Cat, typename Dog>
...
\end{lstlisting}
\end{itemize}


\newpage
\subsection*{Автоматический вывод типа для шаблона функции}
Компилятор может сам догадаться какой тип подставить в шаблонную функцияю по типу аргументов. Если мы в программе вызовем шаблонную функцию \texttt{abs} вот так:
\begin{lstlisting}
abs(-5)
\end{lstlisting}
то компилятор сам догадается, что нужно создать функцию \texttt{abs<int>} и вызвать её в этом месте.
\begin{lstlisting}
#include <iostream>
template<typename T>
T abs(T x)
{
	if (x < 0)
		return -x;
	return x;
}

int main()
{
	int a = -5;
	std::cout << abs<int>(a) << std::endl;  // Можно указать значение параметра шаблона
	std::cout << abs(a) << std::endl;       // Можно не указывать, тогда компилятор сам
	                                        // догадается, что это тип int
}
\end{lstlisting}
\subsection*{Примеры шаблонов функций}
\begin{itemize}
\item Шаблон функции \texttt{getMax}, для нахождения максимума двух объектов.
\begin{lstlisting}
#include <iostream>
#include <string>
using namespace std::string_literals;

template<typename T>
T getMax(const T& a, const T& b) 
{
    if (a > b)
        return a;
    return b;
}

int main() 
{
    std::cout << getMax(5, 10) << std::endl;           // Напечатает 10
    std::cout << getMax(0.2, 0.1) << std::endl;        // Напечатает 0.2
    std::cout << getMax("Cat"s, "Dog"s) << std::endl;  // Напечатает Dog
   
}
\end{lstlisting}
Обратите внимание, что в эту функцию объекты передаются по константной ссылке, а не по значению. Потому что мы не хотим копировать объекты внутрь функции. Ведь эта функция может работать не только с числами, но и с большими объектами, копирование которых внутрь функции может быть неэффективно.

\newpage
\item Шаблон функции \texttt{swap}, для обмена значения двух объектов.
\begin{lstlisting}
#include <iostream>
#include <string>
using namespace std::string_literals;

template<typename T>
void swap(T& a, T& b) 
{
    T temp = a;
    a = b;
    b = temp;
}

int main() 
{
	int a = 10;
	int b = 20;
	swap(a, b);
	std::cout << a << " " << b << std::endl;  // Напечатает 20 10
	
	std::string c = "Cat";
	std::string d = "Dog";
	swap(c, d);
	std::cout << c << " " << d << std::endl;  // Напечатает Dog Cat	
}
\end{lstlisting}


\item Шаблон функции \texttt{sum}, для нахождения суммы элементов вектора.
\begin{lstlisting}
#include <iostream>
#include <string>
#include <vector>
using std::cout, std::endl, std::size_t;

template <typename T>
T sum(const std::vector<T>& v)
{
    T result{};

    for (size_t i = 0; i < v.size(); ++i)
        result += v[i];

    return result;
}

int main() 
{
    std::vector<int> a {10, 20, 30, 40, 50};
    cout << sum(a) << endl;  // Напечатает 150

    std::vector<std::string> b {"Cat", "Dog", "Mouse"};
    cout << sum(b) << endl;  // Напечатает CatDogMouse
}
\end{lstlisting}

\end{itemize}


\subsection*{Ограничения, накладываемые на тип шаблона функции}
Рассмотрим шаблон функции \texttt{abs}.
С какими типами может работать эта шаблонная функция? Можно ли в эту функцию передать, например, объект типа \texttt{std::string}? Если передать в эту функцию строку \texttt{std::string}, то компилятор сгенерирует обычную функцию \texttt{abs<std::string>} из шаблона, подставив везде \texttt{std::string} вместо \texttt{T}. После этого он попытается скомпилировать эту функцию, но у него ничего не получится, так как эта функция использует операторы, которых у класса \texttt{std::string} нет. Ошибка компиляции произойдёт внутри уже сгенерированной функции, когда к объекту типа \texttt{std::string} будет применён оператор, который он не поддерживает.

Всего шаблон \texttt{abs} требует следующих операций:
\begin{enumerate}
\item Оператор \texttt{<} для сравнения с целым числом.
\item Унарный оператор минус (\texttt{-}).
\item Копирование. Так как функция \texttt{abs} принимает аргумент по значению, то этот аргумент должен скопироваться внутрь функции. Поэтому тип \texttt{T} должен поддерживать копирование.
\end{enumerate}
Поскольку у типа \texttt{std::string} отсутствуют первые два необходимых оператора, функция \texttt{abs} не может быть применена к объектам этого типа. Однако для классов, которые поддерживают все требуемые операции, использование функции \texttt{abs} не вызовет никаких проблем.

\begin{lstlisting}
#include <iostream>
#include <string>

template<typename T>
T abs(T x)
{
	if (x < 0)
		return -x;
	return x;
}

struct Alice
{
	int x;
	Alice(int x) : x(x) {}
	
	Alice(const Alice& a) : x(a.x) {}
	bool operator<(int b) const {return x < b;}
	Alice operator-() const
	{
		Alice result(-x);
		return result;
	}
};

int main()
{
	std::string s = "cat";
	std::string t = abs(s);         // Ошибка компиляции
	
	Alice a(-5);            
	Alice b = abs(a);               // OK
	std::cout << b.x << std::endl;  // Напечатает 5
}
\end{lstlisting}



\subsection*{Шаблоны функций от нескольких шаблонных параметров}
Шаблоны могут иметь и несколько параметров.
\begin{lstlisting}
#include <iostream>
#include <string>

template<typename Cat, typename Dog>
Cat func(Dog x)
{
	Cat result(x);
	result += result;
	return result;
}

int main()
{
	double a = func<double, int>(10);  // func вернёт число типа double со значениеи 20.0
	std::string b = func<std::string, const char*>("Hello"); // b будет строкой std::string
													         // со значением HelloHello
}
\end{lstlisting}
Если у шаблона есть несколько параметров, то можно задать только первые несколько параметров, а остальные компилятор попытается вывести. Но не во всех ситуациях вывести тип параметра возможно. Например, нельзя вывести тип, если он используется для только для возвращаемого значения.
\begin{lstlisting}
double a1 = func<double, int>(10);  // Задаём значения для Cat и для Dog

double a2 = func<double>(10);       // Задаём значения для Cat. 
                                    // Компилятор догадается, что Dog должен быть int                  
                                  
double a3 = func(10);               // Компилятор не может знать, чему должно быть равно Cat
                                    // Будет ошибка компиляции.
\end{lstlisting}

\subsection*{Шаблонные аргументы по умолчанию}
\noindent
Также как и для обычных параметров функции, шаблонным параметрам можно дать значение по умолчанию.
\begin{lstlisting}
#include <iostream>

template<typename Cat, typename Dog = char>
void func()
{
	std::cout << sizeof(Cat) << " " << sizeof(Dog) << std::endl;
}

int main()
{
	func<float, double>();  // Напечатает 4 8
	func<float>();          // Напечатает 4 1, тип Dog будет выбран по умолчанию как char
}
\end{lstlisting}

\newpage
\noindent Значение по умолчанию берётся только тогда, когда тип не был указан при вызове и вывести тип по передаваемому аргументу функции нельзя.
\begin{lstlisting}
#include <iostream>

template<typename Cat, typename Dog = short>
void func(Cat c, Dog d)
{
    std::cout << sizeof(c) << " " << sizeof(d) << std::endl;
}

int main()
{
	// В предположении, что размеры char, short, int равны 1, 2, 4 байт соответственно
    int a = 1;
    int b = 1;
    func<char, char>(a, b);  // Напечатает 1 1
    func<char>(a, b);        // Напечатает 1 4
    func(a, b);              // Напечатает 4 4
}
\end{lstlisting}
\subsection*{Шаблоны и перегрузка}
Шаблоны можно использовать вместе с перегрузкой функций.
В случае наличия выбора между шаблоном и точным соответствием перегрузки предпочтение отдается перегрузке
\begin{lstlisting}
#include <iostream>

template<typename T>
void func(T x)
{
    std::cout << "Template" << std::endl;
}

void func(int x)
{
	std::cout << "Int" << std::endl;
}

int main()
{
    func(10);     // Напечатает Int
    func(1.5);    // Напечатает Template
    func("Cat");  // Напечатает Template
}
\end{lstlisting}

\newpage
\subsection*{Шаблоны с нетиповыми параметрами}
Шаблонными аргументами могут быть не только типы но и объекты некоторых типов, примерно также как и обычные аргументы функций. Только в отличии от значений обычных аргументов, значения шаблонных аргументов должны быть известны на этапе компиляции.
\begin{lstlisting}
#include <iostream>

template<int A, int B>
int sum() 
{
    return A + B;
}

int main() 
{
    std::cout << sum<10, 20>() << std::endl;  // Напечатает 30
    
    int a, b;
    std::cin >> a >> b;
    int c = sum<a, b>();  // Ошибка. Аргументы шаблона должны быть известны на этапе компиляции
}
\end{lstlisting}
Такие шаблонные параметры раскрываются также, как и типовые параметры шаблонов. То есть, когда компилятор видит в коде \lstinline|sum<10, 20>()| он по шаблонной функции \texttt{sum} создаёт обычную функцию \texttt{sum<10, 20>} заменяя \texttt{A} на \texttt{10}, а \texttt{B} на \texttt{20}. Значения аргументов шаблонной функции должны быть известны на этапе компиляции, так как генерация функции из шаблона и подстановка всех значений происходит на этапе компиляции.

Далеко не каждый тип может быть типом шаблонного параметра. Например, \texttt{std::string}  не может быть типом шаблонного параметра, то есть нельзя написать так:
\begin{lstlisting}
template<std::string A>
A func() 
{
    return A;
}
\end{lstlisting}
В качестве параметров можно использовать целые числа и числа с плавающей точкой и некоторые другие типы. Точные правила какие типы можно использовать для шаблонных параметров, а какие нет достаточно сложны и не будут тут приводиться.

Вообще, нетиповые параметры используются реже, чем типовые(\texttt{typename}) параметры. Когда же такие параметры используются, то обычно используются целочисленные типы, такие как \texttt{int} или \texttt{size\_t}.

\subsection*{Два этапа компиляции шаблона}
Рассмотрим следующую программу, содержащую шаблон \texttt{func}:
\begin{lstlisting}
template<typename T>
int func() 
{
    T x;
    x.abracadabra();
}

int main() {}
\end{lstlisting}

Функция \texttt{func} создаёт объект типа \texttt{T} и вызывает у этого объекта метод \texttt{abracadabra}. При этом ни одного класса с методом \texttt{abracadabra} у нас нет. Скомпилируется ли эта программа?



\subsection*{Вывод типа в шаблонных функциях по аргументу}
В \texttt{seminar4\_templates/theory/code/template\_type\_deduction/get\_type\_name.hpp} лежит функция \texttt{getTypeName}, которая принимает на вход (через шаблонный параметр) тип и возвращает строковое представление данного типа. Эта функция будет очень полезна при анализировании того, какие типы выводятся в шаблонах. Помимо этой функции будет использоваться оператор \texttt{decltype}, который просто возвращает тип выражения.

Использовать функцию \texttt{getTypeName} можно следующим образом:
\begin{lstlisting}
#include <iostream>
#include "get_type_name.hpp"
int main()
{
    int a = 10;

    std::cout << getTypeName<int>() << std::endl;
    std::cout << getTypeName<decltype(a)>() << std::endl;
}
\end{lstlisting}

\subsubsection*{Шаблонная функция принимает по значению}
Рассмотрим случаи автоматического вывода шаблонного типа, при котором получаемый тип будет не соответствовать типу аргумента функции. Для начала, случай когда шаблонная функция принимает по значению. В этом случае, итоговый тип будет являться decay-версией типа аргумента. Всего есть три правила для распада типа:
\begin{enumerate}
\item Ссылки и \texttt{const}-квалификаторы отбрасываются.
\item Массивы в стиле C преобразуются в указатели (array to pointer decay).
\item Функции преобразуются в указатели на функции.
\end{enumerate}
Рассмотрим пример, иллюстрирующий такое поведение:
\begin{lstlisting}
#include <iostream>
#include "get_type_name.hpp"
template<typename T>
void func(T x)
{
    std::cout << getTypeName<T>() << std::endl;
}

int main()
{
    int a = 10;
    func(a);              	// int, так как тип выведется как int
    func<int&>(a);			// int\&, если мы указываем тип, то он не выводится
    func<const int&>(a);    // const int\&, если мы указываем тип, то он не выводится

    int& b = a;
    func(b);				// int, так как амперсанд отбрасывается
    const int& c = a;    
    func(c);				// int, так как const и амперсанд отбрасывается

    int d[3] = {10, 20, 30};
    func(d);				// int*, так как массив разлагается в указатель
    func("Hello");
}
\end{lstlisting}


\subsubsection*{Шаблонная функция принимает по ссылке или \texttt{const} ссылке}
Если же шаблонная функция принимает по ссылке (обычной или константной), то никакого разложения происходить не будет. Тип \texttt{T} будет подбираться таким образом, чтобы функция смогла принять передаваемый аргумент.

\begin{lstlisting}
#include <iostream>
#include "get_type_name.hpp"

template<typename T>
void func(T& x)
{
    std::cout << getTypeName<decltype(x)>() << std::endl;
}

int main()
{
    int a = 10;
    func(a);				// int\&
    func<int&>(a);			// int\&
    func<const int&>(a);	// int\&

    const int& b = a;		// const int\&
    func(b);

    int arr[3] = {10, 20, 30};	
    func(arr);				// int(\&)[3]  (ссылка на массив)
}
\end{lstlisting}



\newpage
\section*{Ключевое слово \texttt{auto}}
Ключевое слово \texttt{auto} используется для автоматического вывода типа.
\begin{lstlisting}
#include <string>
int main() 
{
    auto a = 123;   // a будет иметь тип int
    auto b = 4.1;   // b будет иметь тип double
    auto c = 4.1f;  // c будет иметь тип float
    
    auto s1 = "Hello";               // s1 будет иметь тип const char*
    auto s2 = std::string("Hello");  // s2 будет иметь тип std::string
}
\end{lstlisting}


\subsection*{Вывод типа при использовании \texttt{auto}}
Правила вывода типа при использовании ключевого слова \texttt{auto} такие же, что и при выводе типов шаблонных параметров.
\begin{lstlisting}
#include <iostream>
#include "get_type_name.hpp"

void f(int x) {}

int main()
{
    auto a = 10;
    std::cout << getTypeName<decltype(a)>() << std::endl;

    int& r = a;
    auto b = r;
    std::cout << getTypeName<decltype(b)>() << std::endl;

    const int& cr = a;
    auto c = cr;
    std::cout << getTypeName<decltype(c)>() << std::endl;


    int arr[3] = {10, 20, 30};
    auto d = arr;
    std::cout << getTypeName<decltype(d)>() << std::endl;


    std::cout << "---\n";
    std::cout << getTypeName<decltype(f)>() << std::endl;
    auto e = f;
    std::cout << getTypeName<decltype(e)>() << std::endl;
}
\end{lstlisting}





\subsection*{\texttt{auto} и возвращаемое значение функции}
trailing return type. хвостовой тип возвращаемого значения.
\begin{lstlisting}
auto factorial(int n) 
{
    if (n > 0)
        return n * factorial(n - 1);
    return 1;
}
\end{lstlisting}

\iffalse
\subsection*{Задачи:}
\begin{itemize}
\item Протестируйте, можно ли использовать \texttt{auto} вместо возвращаемого типа функции. Напишите функцию, которая принимает на вход вектор строк и возвращает строку, которая является результатом конкатенации всех строк. Вместо возвращаемого типа используйте \texttt{auto}.
\item Протестируйте, можно ли создать функцию, которая будет принимать целое число и, в зависимости от этого числа, возвращать значения разных типов. (Если вместо возвращаемого типа используется \texttt{auto}).

\item Протестируйте, можно ли использовать о указатель с помощью \texttt{auto}. Пусть есть такой участок кода:
\begin{lstlisting}
int a = 123;
auto p = &a;
auto* q = &a;
\end{lstlisting}
Какой тип будет у \texttt{p} и \texttt{q}?
\item Функция вычисления факториала, написанная ниже с использованием \texttt{auto} не работает. 
\begin{lstlisting}
auto factorial(int n) 
{
    if (n > 0)
        return n * factorial(n - 1);
    return 1;
}
\end{lstlisting}
Почему? Исправьте эту функцию, не убирая \texttt{auto}.
\end{itemize}
\fi


\subsection*{\texttt{auto} и шаблоны}
Ключевое слово \texttt{auto} можно использовать не только при объявлении новых переменных, но и для параметров функций. То есть можно писать вот так:
\begin{lstlisting}
void func(auto x)
{
	...
}
\end{lstlisting}
В этом случае компилятор будет воспринимать эту функцию не как обычную, а как шаблонную функцию следующего вида:
\begin{lstlisting}
template<typename T>
void func(T x)
{
	...
}
\end{lstlisting}
Приведём примеры функций, которые используют \texttt{auto} для типа параметра:
\begin{itemize}
\item Функция, которая печатает размер объекта в байтах:
\begin{lstlisting}
#include <iostream>

void printSize(auto x)
{
	std::cout << sizeof(x) << std::endl;
}

int main()
{
	printSize(10);
	printSize(1.5);
}
\end{lstlisting}

\item Функция нахождения модуля числа:
\begin{lstlisting}
#include <iostream>

auto abs(auto x)
{
	if (x < 0)
		return -x;
	return x;
}

int main()
{
	std::cout << abs<int>(-5) << std::endl;       // Напечатает 5
	std::cout << abs<double>(-1.5) << std::endl;  // Напечатает 1.5
}
\end{lstlisting}
\end{itemize}

\subsection*{Range-based циклы}
Циклы, основанные на диапазоне, предоставляют более простой способ обхода контейнера:
\begin{lstlisting}
#include <iostream>
#include <vector>

int main()
{
    std::vector<int> v {6, 1, 7, 4};
    for (int num : v)
    	std::cout << num << std::endl;    
}
\end{lstlisting}
Для изменения элементов контейнера при обходе нужно использовать ссылки:
\begin{lstlisting}
for (int& num : v)
    num += 1;    
\end{lstlisting}



\subsection*{Structure binding (структурное связывание)}
Для объявления и инициализации нескольких переменных через агрегатный объект можно использовать так называемое структурное связывания. В коде ниже мы объявляем переменные \texttt{a} и \texttt{b} одной строкой с помощью структурного связывания.
\begin{lstlisting}
#include <iostream>
struct Point
{
	int x;
	int y;
};
int main() 
{
    Point p {5, 1};
    auto [a, b] = p;
    
    std::cout << a << " " << b << std::endl;
}
\end{lstlisting}



\newpage
\section*{Шаблоны классов}
\begin{lstlisting}
#include <iostream>
#include <string>
#include <cstdlib>
using std::cout, std::endl, std::size_t;

template <typename T, size_t Size>
class Array 
{
private:
    T data[Size];

public:
    T& operator[](size_t id) 
    {
        return data[id];
    }
};

template <typename T, size_t Size>
std::ostream& operator<<(std::ostream& out, Array<T, Size> array) 
{
    out << "[";
    for (size_t i = 0; i < Size - 1; ++i)
        out << array[i] << ", ";
    
    if (Size != 0)
        out << array[Size - 1];

    out << "]"; 
    return out;
}

int main() 
{
    Array<int, 10> a;
    for (int i = 0; i < 10; ++i)
    {
        a[i] = rand() % 100;
    }
    cout << a << endl;

    Array<std::string, 5> b;
    b[0] = "Cat";
    b[1] = "Dog";
    b[2] = "Zebra";
    b[3] = "Elephant";
    b[4] = "Whale";
    cout << b << endl;
}
\end{lstlisting}




\subsection*{Зависимые имена}
Можно создавать новые имена в классах:
\begin{lstlisting}
#include <iostream>
struct Alice 
{
	typedef int cat;
	using dog = int;
};

int main()
{
	std::cout << sizeof(Alice::cat) << std::endl;
	std::cout << sizeof(Alice::dog) << std::endl;
}
\end{lstlisting}
Но есть неоднозначность между новым именем и статической переменной.
\begin{lstlisting}
#include <iostream>

struct Alice 
{
	using cat = int;
};

template<typename T>
struct Bob 
{
	void func()
	{
		T::cat* x;
	}
};

int main()
{
	Bob<Alice> b;
	b.func();
}
\end{lstlisting}


\subsection*{Зависимый шаблон}

\begin{lstlisting}
#include <iostream>
struct Alice
{
    template<int A>
    void get() 
    {
        std::cout << "get method" << std::endl;
    }
};

struct Bob
{
    int get {10};
};

template <typename T>
void func(T x) 
{
    //x.get<0>();        // Неправильно, так как get будет считаться полем
    x.template get<0>(); // Правильно, теперь get это шаблонный метод
}

int main()
{
    Alice a;
    func(a);
    // Bob b;
    // func(b);
}
\end{lstlisting}

\subsection*{Специализация}
\begin{lstlisting}
template<typename T>
struct Cat {
    static void f() { std::cout << "primary\n"; }
};

template<>
struct Cat<int> {
    static void f() { std::cout << "specialized for int\n"; }
};
\end{lstlisting}


\section*{Шаблоны новых имён типов}
\begin{lstlisting}
#include <vector>

template<typename T>
using MyType = std::vector<T>;

int main()
{
	MyType a = {10, 20, 30, 40, 50};
}
\end{lstlisting}



\section*{Примеры шаблонов в стандартной библиотеке C++}
\subsection*{Класс \texttt{std::vector}}
\subsection*{Класс \texttt{std::array}}
\subsection*{Класс \texttt{std::pair}}
\subsection*{Класс \texttt{std::optional}}
\subsection*{Класс \texttt{std::basic\_string}}
\texttt{std::basic\_string} — это базовый шаблон, на котором построены все обычные строки в C++:

\begin{itemize}
\item \texttt{std::string = std::basic\_string<char>}
\item \texttt{std::wstring = std::basic\_string<wchar\_t>}
\item и т. д.
\end{itemize}



\section*{Вариативные шаблоны}
Вариативные шаблоны (variadic templates) -- это механизм, позволяющий шаблонам принимать произвольное число параметров.

\begin{lstlisting}
template<typename... Types>
void func(Types... args) {}
\end{lstlisting}

Использование пака типов и пака параметров:
\begin{lstlisting}
template<typename... Types>
void func(Types... args) 
{
	// Types -- особая сущность, пак типов
	// args  -- особая сущность, пак аргументов
	
	// Что можно сделать с паками:
	
	f<Types...>() - развернуть пак типов
	f(args...)    - развернуть тип аргументов
	
	sizeof...(args)	 - количество аргументов в паке
	sizeof...(Types) - количество аргументов в паке
	
	// fold expressions
}
\end{lstlisting}

\subsection*{Рекурсивное раскрытие параметров}

\begin{lstlisting}
void print() {}

template<typename T, typename... Ts>
void print(T first, Ts... rest) 
{
    std::cout << first << "\n";
    print(rest...);
}
\end{lstlisting}


\subsection*{Fold expressions}

\begin{lstlisting}
template<typename... Ts>
auto sum(Ts... ts) 
{
    return (ts + ...);
}


template<typename... Ts>
auto sumsq(Ts... ts) 
{
    return ((ts * ts) + ...);
}


template<typename T>
void print_one(const T& t) {
    std::cout << t << "\n";
}

template<typename... Ts>
void print_all(const Ts&... ts) {
    (print_one(ts), ...);
}
\end{lstlisting}


\end{document}
