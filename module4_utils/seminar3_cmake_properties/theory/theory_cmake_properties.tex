\documentclass{article}
\usepackage[english,russian]{babel}
\usepackage{textcomp}
\usepackage{geometry}
  \geometry{left=2cm}
  \geometry{right=1.5cm}
  \geometry{top=1.5cm}
  \geometry{bottom=2cm}
\usepackage{tikz}
\usepackage{multicol}
\usepackage{hyperref}
\usepackage{listings}
\pagenumbering{gobble}

\lstdefinestyle{csMiptCppStyle}{
  language=C++,
  basicstyle=\linespread{1.1}\ttfamily,
  columns=fixed,
  fontadjust=true,
  basewidth=0.5em,
  keywordstyle=\color{blue}\bfseries,
  commentstyle=\color{gray},
  texcl=true,
  stringstyle=\ttfamily\color{orange!50!black},
  showstringspaces=false,
  numbersep=5pt,
  numberstyle=\tiny\color{black},
  numberfirstline=true,
  stepnumber=1,      
  numbersep=10pt,
  backgroundcolor=\color{white},
  showstringspaces=false,
  captionpos=b,
  breaklines=true
  breakatwhitespace=true,
  xleftmargin=.2in,
  extendedchars=\true,
  keepspaces = true,
  tabsize=4,
  upquote=true,
}


\lstdefinestyle{csMiptCppLinesStyle}{
  style=csMiptCppStyle,
  frame=lines,
}

\lstdefinestyle{csMiptCppBorderStyle}{
  style=csMiptCppStyle,
  framexleftmargin=5mm, 
  frame=shadowbox, 
  rulesepcolor=\color{gray}
}


\lstdefinestyle{csMiptBash}{
breaklines=true,
frame=tb,
language=bash,
breakatwhitespace=true,
alsoletter={*()"'0123456789.},
alsoother={\{\=\}},
basicstyle={\ttfamily},
keywordstyle={\bfseries},
literate={{=}{{{=}}}1},
prebreak={\textbackslash},
sensitive=true,
stepnumber=1,
tabsize=4,
morekeywords={echo, function},
otherkeywords={-, \{, \}},
literate={\$\{}{{{{\bfseries{}\$\{}}}}2,
upquote=true,
frame=none
}




\lstset{style=csMiptCppLinesStyle}
\lstset{literate={~}{{\raisebox{0.5ex}{\texttildelow}}}{1}}


\renewcommand{\thesection}{\arabic{section}}
\makeatletter
\def\@seccntformat#1{\@ifundefined{#1@cntformat}%
   {\csname the#1\endcsname\quad}%    default
   {\csname #1@cntformat\endcsname}}% enable individual control
\newcommand\section@cntformat{Часть \thesection:\space}
\makeatother




\begin{document}
\title{Семинар \#3: CMake \\[1ex] \large Переменные. Свойства. Конфигурации. Генераторные выражения. \vspace{-5ex}}
\date{}\maketitle

\noindent Несколько исполняемых файлов в проекте.\\
\texttt{cmake\_minimum\_required} - policy\\
\texttt{target\_sources}\\
Многострочные строки. [==[  ]==].\\
if no scope.\\
if defined, if exists, if matches\\
math.\\
\texttt{cmake\_parse\_arguments}\\
модули include\\
области видимости в модулях и поддиректориях.
\section{Переменные}

\subsection*{Переменные среды}

\subsection*{Кешированные переменные}

\subsection*{Стандартные переменные CMake}


\section{Поддиректории и модули}

\section{Свойства}

\section{Типы сборки (конфигурации)}
\subsection*{5 типов типов сборки по умолчанию}
\subsection*{Одноконфигурационные и многоконфигурационные генераторы}

\subsection*{Выбор типа сборки для одноконфигурационного генератора}
Кэшированная переменная \texttt{CMAKE\_BUILD\_TYPE}.\\
Кэшированные переменные и свойства, соответствующие конфигурациям.\\


\subsection*{Выбор типа сборки для многоконфигурационного генератора}
Глобальное свойство \texttt{GENERATOR\_IS\_MULTI\_CONFIG}\\
Кэшированная переменная \texttt{CMAKE\_CONFIGURATION\_TYPES}.\\
Опция программы cmake \texttt{--config}.\\


\subsection*{Создание своего типа сборки}
\begin{itemize}
\item \texttt{CMAKE\_C\_FLAGS\_<CONFIG>}
\item \texttt{CMAKE\_CXX\_FLAGS\_<CONFIG>}
\item \texttt{CMAKE\_EXE\_LINKER\_FLAGS\_<CONFIG>}
\item \texttt{CMAKE\_SHARED\_LINKER\_FLAGS\_<CONFIG>}
\item \texttt{CMAKE\_STATIC\_LINKER\_FLAGS\_<CONFIG>}
\item \texttt{CMAKE\_MODULE\_LINKER\_FLAGS\_<CONFIG>}
\item \texttt{CMAKE\_<CONFIG>\_POSTFIX}
\end{itemize}



\section{Выполнение на этапе компиляции}

\section{Генераторные выражения}

\end{document}
