\documentclass{article}
\usepackage[utf8x]{inputenc}
\usepackage{ucs}
\usepackage{amsmath} 
\usepackage{amsfonts}
\usepackage{marvosym}
\usepackage{wasysym}
\usepackage{upgreek}
\usepackage[english,russian]{babel}
\usepackage{graphicx}
\usepackage{float}
\usepackage{textcomp}
\usepackage{hyperref}
\usepackage{geometry}
  \geometry{left=2cm}
  \geometry{right=2cm}
  \geometry{top=2cm}
  \geometry{bottom=2cm}
\usepackage{tikz}
\usepackage{ccaption}
\usepackage{multicol}
\usepackage{fancyvrb}
\usepackage{xcolor}

\hypersetup{
   colorlinks=true,
   citecolor=blue,
   linkcolor=black,
   urlcolor=blue
}

\usepackage{listings}
%\setlength{\columnsep}{1.5cm}
%\setlength{\columnseprule}{0.2pt}

\usepackage[absolute]{textpos}


\usepackage{colortbl,graphicx,tikz}
\definecolor{X}{rgb}{.5,.5,.5}

\renewcommand{\thesubsection}{\arabic{subsection}}

\begin{document}
\pagenumbering{gobble}
\lstset{
  language=C++,                % choose the language of the code
  basicstyle=\linespread{1.1}\ttfamily,
  columns=fixed,
  fontadjust=true,
  basewidth=0.5em,
  keywordstyle=\color{blue}\bfseries,
  commentstyle=\color{gray},
  stringstyle=\ttfamily\color{orange!50!black},
  showstringspaces=false,
  numbersep=5pt,
  numberstyle=\tiny\color{black},
  numberfirstline=true,
  stepnumber=1,                   % the step between two line-numbers.        
  numbersep=10pt,                  % how far the line-numbers are from the code
  backgroundcolor=\color{white},  % choose the background color. You must add \usepackage{color}
  showstringspaces=false,         % underline spaces within strings
  captionpos=b,                   % sets the caption-position to bottom
  breaklines=true,                % sets automatic line breaking
  breakatwhitespace=true,         % sets if automatic breaks should only happen at whitespace
  xleftmargin=.2in,
  extendedchars=\true,
  keepspaces = true,
}
\lstset{literate=%
   *{0}{{{\color{red!20!violet}0}}}1
    {1}{{{\color{red!20!violet}1}}}1
    {2}{{{\color{red!20!violet}2}}}1
    {3}{{{\color{red!20!violet}3}}}1
    {4}{{{\color{red!20!violet}4}}}1
    {5}{{{\color{red!20!violet}5}}}1
    {6}{{{\color{red!20!violet}6}}}1
    {7}{{{\color{red!20!violet}7}}}1
    {8}{{{\color{red!20!violet}8}}}1
    {9}{{{\color{red!20!violet}9}}}1
}
\newcommand\upquote[1]{\textquotesingle#1\textquotesingle}

\renewcommand{\thesubsection}{\arabic{subsection}}
\makeatletter
\def\@seccntformat#1{\@ifundefined{#1@cntformat}%
   {\csname the#1\endcsname\quad}%    default
   {\csname #1@cntformat\endcsname}}% enable individual control
\newcommand\section@cntformat{}     % section level 
\newcommand\subsection@cntformat{Задача \thesubsection.\space} % subsection level
\newcommand\subsubsection@cntformat{\thesubsubsection.\space} % subsubsection level
\makeatother


\makeatletter
\newcount\my@repeat@count
\newcommand{\myrepeat}[2]{%
  \begingroup
  \my@repeat@count=\z@
  \@whilenum\my@repeat@count<#1\do{#2\advance\my@repeat@count\@ne}%
  \endgroup
}
\makeatother

\title{Семинар: Git. Домашнее задание.\vspace{-5ex}}\date{}\maketitle
\subsection{Простой проект}
В папке \texttt{image\_project} содержится исходный код простой программы, которая создаёт изображение.
Можно скомпилировать эту программу командой:
\begin{verbatim}
g++ main.cpp image.cpp
\end{verbatim}
Если запустить получившийся исполняемый файл, то программа создаст изображение \texttt{result.ppm}.
Открыть изображение такого формата можно любой продвинутой программой для просмотра изображений, например, можно использовать IrfanView.

\begin{enumerate}
\item Создайте git-репозиторий для этого проекта, используя \texttt{git init}.
\item Сделайте первый коммит в этот репозиторий. Коммит должен содержать все файлы исходного кода проекта (\texttt{image.h}, \texttt{image.cpp} и \texttt{main.cpp}).
\item Измените файл \texttt{main.cpp} так, чтобы на изображении рисовался ещё один круг. Добавьте это изменение в репозиторий, сделав ещё один коммит.
\item Добавьте в класс \texttt{Image} метод \texttt{getNumberOfPixels}, который будет возвращать количество пикселей изображения. Сделайте соответствующие изменения в файлы \texttt{image.cpp} и \texttt{image.h} и добавьте эти изменения в репозиторий.
\item Добавьте в этот проект поддержку системы сборки Cmake. Добавьте файл \texttt{CMakeLists.txt} и соберите файл в новой папке \texttt{build}. Добавьте изменения в репозиторий. Коммит должен состоять только из одного файла (\texttt{CMakeLists.txt}). Содержимое папки \texttt{build} и прочие файл добавлять в репозиторий не надо.
\item Создайте файл \texttt{.gitignore} и добавьте туда правила для игнорируемых файлов. Система контроля версий должна игнорировать папку \texttt{build}. Добавьте файл \texttt{.gitignore} в репозиторий, сделав ещё один коммит.
\item Выполните команду \texttt{git log} чтобы увидеть информацию о ваших коммитах.

\item Класс \texttt{Image} может считывать изображения в формате ppm. Хочется добавить в проект изображение и изменить файл \texttt{main.cpp} так, чтобы программа считывала это изображение. Однако, также не хочется терять тот код, который сейчас находится в файле \texttt{main.cpp}. \\
Поэтому создадим новую побочную ветку, чтобы вносить изменения в неё. Используйте \texttt{git branch} и создайте ветку под названием \texttt{ppm}. 

\item Выполните команду \texttt{git branch} чтобы посмотреть на ваши ветки. В данный момент у вас должно быть 2 ветки \texttt{main} и \texttt{ppm} (главная ветка по умолчанию может называться \texttt{main} или \texttt{master}). Та ветка на которой вы находитесь должна быть помечена звёздочкой. Вы должны находиться на ветке \texttt{ppm}. Если это не так, то перейдите на неё с помощью \texttt{git switch}.

\item Создайте новую папку \texttt{ppm\_examples} и скопируйте в неё файл \texttt{files/zlatoust1910.ppm}. Измените файл \texttt{main.cpp} так, чтобы он считывал этот файл, рисовал на нём кружок и сохранял результат в файл \texttt{result.ppm}. Сохраните изменения, сделав коммит в ветку \texttt{ppm}. 

\item Вернитесь обратно на ветку \texttt{main} с помощью \texttt{git switch}. Изменения, сделанные в предыдущем пункте должны исчезнуть (их можно вернуть, если снова перейти на ветку \texttt{ppm}).

\item Теперь хочется добавить поддержку изображений в формате jpg. Эти изменения нужно сделать в новой ветке под названием \texttt{jpg}. Создайте эту ветку и перейдите на неё.

\item Для загрузки изображений в формате \texttt{jpg} будем использовать библиотеку stb. Эта библиотека и пример её использования лежит в папке \texttt{files/stb}. Скопируйте файлы из этой папки в папку проекта. Измените класс \texttt{Image} так, чтобы он умел считывать файл в формате \texttt{jpg} (используйте библиотеку stb). Скопируйте файл \texttt{files/stb/zlatoust1910.jpg} в папку \texttt{jpg\_examples} проекта. Сохраните все эти изменения в ветке \texttt{jpg}.

\item Вернитесь обратно на ветку \texttt{main} с помощью \texttt{git switch}. Изменения, сделанные в предыдущем пункте должны исчезнуть (их можно вернуть, если снова перейти на ветку \texttt{jpg}).

\item Добавьте метод \texttt{convertToGrayscale} в класс \texttt{Image}. Этот метод должен делать всё изображение чёрно-белым. Чтобы это сделать, усредните компоненты цвета каждого пикселя. Протестируйте работу этого метода в файле \texttt{main.cpp}. (Сделайте черно-белой картинку с тремя отрезками и двумя кругами). Сохраните изменения в репозиторий, сделав коммит в ветку \texttt{main}.

\item Посмотрите на граф, который образуют ваши ветки, используя:
\begin{verbatim}
git log --oneline --graph --all
\end{verbatim}

\item Теперь нужно добавить метод \texttt{convertToGrayscale} в ветку \texttt{ppm}. Для этого перейдите в ветку \texttt{ppm} с помощью \texttt{git switch} и используйте \texttt{git merge} чтобы добавить изменения из ветки \texttt{main} в ветку \texttt{ppm}.

\item Также нужно добавить метод \texttt{convertToGrayscale} в ветку \texttt{jpg}. Но теперь будем использовать \texttt{rebase}. Для этого перейдите в ветку \texttt{jpg} с помощью \texttt{git switch} и используйте \texttt{git rebase} чтобы добавить изменения из ветки \texttt{main} в ветку \texttt{ppm}.

\item Посмотрите на граф, который образуют ваши ветки, используя:
\begin{verbatim}
git log --oneline --graph --all
\end{verbatim}

\item Соедините теперь все 3 ветки (\texttt{main}, \texttt{ppm} и \texttt{jpg}) в одну ветку \texttt{main}.
Итоговый класс \texttt{Image} должен уметь работать с файлами в форматах \texttt{ppm} и \texttt{jpg}. В файле \texttt{main.cpp} нужно считывать/изменять/сохранять как \texttt{ppm} изображение, так и \texttt{jpg} изображения.

\item Удалите ветки \texttt{ppm} и \texttt{jpg}.

\item Посмотрите на граф, который образуют ваши ветки, используя:
\begin{verbatim}
git log --oneline --graph --all
\end{verbatim}

\item Чтобы сдать задание нужно загрузить локальный репозиторий этого проекта на гитхаб. Для этого нужно сделать следуюшее:

\begin{enumerate}
\item Создать новый пустой репозиторий на гитхабе
\item Добавить ссылку на удалённый репозиторий в ваш локальный репозиторий с помощью команды:
\begin{verbatim}
git remote add origin "URL вашего нового репозитория на гитхаб"
\end{verbatim}

\item Обновить репозиторий на гитхаб:
\begin{verbatim}
git push -u origin main
\end{verbatim}
\end{enumerate}

\end{enumerate}



\end{document}
